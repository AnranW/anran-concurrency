\documentclass[adraft]{eptcs}
\usepackage{breakurl}
\usepackage{underscore}
\usepackage{amssymb}
\usepackage{latexsym}
\usepackage{wrapfig}
\usepackage{amsmath}





%%%%%%%%%%%%%%%%%%%%%%%%%%%%%%%%%%%%%%%%%%%%%%%%%%%%%%%%%%%%%%%%%%%%%%%%%%%%%%
\begin{document}

\title{Completion Report for RISE-Worldwide Program of DAAD}
\def\titlerunning{Report}
\author{Anran Wang}
\def\authorrunning{Anran Wang}
\maketitle

\section{General}
From 6th August, 2018 to 12th October, 2018 I participated in the RISE-Worldwide internship program funded by DAAD. The internship was supervised by Dr. Peter H\"ofner and Dr. Robert van Glabbeek, CSIRO, and took place in the University of New South Wales, Sydney, Australia.

\subsection{Vaccination}
No vaccination was needed for me to get into Australia.

\subsection{Visa Application}
I applied for the visa 408 (temporary working visa), which was recommended by my supervisors, allowing me to stay in Australia for up to 90 days doing my internship. The visa application was quite standard, I needed to provide information about my passport, funding, visa for Germany (because I am not a German citizen), etc. A special requirement was to provide the company's business code, which could be found in the contract that I signed with CSIRO. I was issued with an electronic visa, I printed it out just in case, but the paper version did not came into use. My passport is an e-passport, which means it has a chip that can be read when getting into and out of the Australian border, avoiding a long queue.

\subsection{Travel Preparation}
I stayed ten weeks in Sydney, packed everything in a small suitcase and a backpack, and did not buy anything other than food, toiletries, and pair of shoes (because mine broke down after long walks up and down the coast). Sydney is a big city, and it was quite convenient to buy anything when it comes into need.

The weather from August to October in Sydney was pretty nice, I brought an umbrella but barely used it. A raincoat or water-proofing jumper would also have sufficed. It did get windy when it started to rain.

Finding an accommodation in Sydney seemed a little bit easier than in Germany, there were a number of vacant houses. Before coming to Australia, I booked a room online for a week for temporary stay and managed to find a satisfying room after five days. The rent is paid per week and I paid 400 AUD each week, staying near the beach. The price varies, depending on where you are staying, but one can expect to pay for around 300 dollars to get a decent room. I was asked to pay a bond with the value of two weeks' rent.

Before getting to Australia, I did not manage to get a lot of cash, I only got 100 AUD with face value of 50 dollars, and it caused problems when I was trying to get to my accommodation from the airport by bus, because the driver did not have enough change. It is worth of recommending to directly buy an opal card for adults (which is the card for paying public transports in the state of New South Wales) in the airport, because the cards themselves do not cost money, and I was not eligible for a concession card.

I booked my trip roughly two months ahead, and it costed around 1400 EUR, which is the same amount that is paid by DAAD. The flight is around 26 hours, and has officially one transfer in Dubai, but from Dubai to Sydney, the flight makes a stop at Bangkok, and passengers had to get out of the plane and go through security check again, so it is indeed two transfers, which was not clearly stated in the flight ticket. The flight ticket comes with two complementary Deutsche Bahn tickets, which could be used to get to and leave the airports.

During my internship, I did not apply for an Australian bank card, and I used the card from Deutsche Bank, who has a partner bank called Westpac, and I could use the ATM of Westpac to withdraw cash from my Deutsche Bank card with fair exchange rate. This is quite a good option.


%%%%%%%%%%%%%%%%%%%%%%%%%%%%%%%%%%%%
\section{Professional}
\subsection{The Lab}
The lab where my internship is carried out is called Data61, which is a lab of CSIRO. The offices are in Computer Science building of the University of New South Wales, right in the university area. The people in the lab are really friendly, and always willing to help. I volunteered for three weeks to look after the fish tank in the office, which was a lot of fun.

\subsection{The Internship}
The project I signed for is \emph{Expressiveness}, where the power of expressing systems of different process algebras are compared. But after two weeks, I found interest in another project, which is focused on building a theory for \emph{Justness}. My supervisors and another co-intern Filippo De Bortoli are working on the project, and I followed their steps to learn more. After roughly five weeks, I finished some proofs and thought that I wrapped the part I was looking at, which was proved untrue with a counterexample.

During the last three weeks, I went back to the \emph{Expressiveness} project and looked at an encoding from ABC(Algebra of Broadcast Communication) to CCS$_P$ (Calculus of Communicating Systems with Priority) proposed by Johannes \AA man Pohjola, and tried to prove that the encoding is good enough according to some properties.

\subsection{What was learned}
I did only a little preparation before coming to Sydney, which made the beginning of my internship particularly hard. The topic of process algebra itself is complex, and it takes a lot of time to begin to catch up. Enough preparation is going to be helpful.

My supervisors are sometimes not available, and I found it really helpful to talk to other interns as well as other scholars in the lab, even if we are not learning about the same project.

The experience is overall very pleasant, I had a great time in Sydney and got to know a lot of nice people, and also learned a lot. I am very thankful to my supervisors, my co-interns and people in the lab, as well as DAAD.


\end{document}
